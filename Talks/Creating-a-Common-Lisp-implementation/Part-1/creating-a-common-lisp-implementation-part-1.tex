\documentclass{beamer}
\usepackage[utf8]{inputenc}
\beamertemplateshadingbackground{red!10}{blue!10}
%\usepackage{fancybox}
\usepackage{epsfig}
\usepackage{verbatim}
\usepackage{url}
%\usepackage{graphics}
%\usepackage{xcolor}
\usepackage{fancybox}
\usepackage{moreverb}
%\usepackage[all]{xy}
\usepackage{listings}
\usepackage{filecontents}
\usepackage{graphicx}

\lstset{
  language=Lisp,
  basicstyle=\scriptsize\ttfamily,
  keywordstyle={},
  commentstyle={},
  stringstyle={}}

\def\inputfig#1{\input #1}
\def\inputeps#1{\includegraphics{#1}}
\def\inputtex#1{\input #1}

\inputtex{logos.tex}

%\definecolor{ORANGE}{named}{Orange}

\definecolor{GREEN}{rgb}{0,0.8,0}
\definecolor{YELLOW}{rgb}{1,1,0}
\definecolor{ORANGE}{rgb}{1,0.647,0}
\definecolor{PURPLE}{rgb}{0.627,0.126,0.941}
\definecolor{PURPLE}{named}{purple}
\definecolor{PINK}{rgb}{1,0.412,0.706}
\definecolor{WHEAT}{rgb}{1,0.8,0.6}
\definecolor{BLUE}{rgb}{0,0,1}
\definecolor{GRAY}{named}{gray}
\definecolor{CYAN}{named}{cyan}

\newcommand{\orchid}[1]{\textcolor{Orchid}{#1}}
\newcommand{\defun}[1]{\orchid{#1}}

\newcommand{\BROWN}[1]{\textcolor{BROWN}{#1}}
\newcommand{\RED}[1]{\textcolor{red}{#1}}
\newcommand{\YELLOW}[1]{\textcolor{YELLOW}{#1}}
\newcommand{\PINK}[1]{\textcolor{PINK}{#1}}
\newcommand{\WHEAT}[1]{\textcolor{wheat}{#1}}
\newcommand{\GREEN}[1]{\textcolor{GREEN}{#1}}
\newcommand{\PURPLE}[1]{\textcolor{PURPLE}{#1}}
\newcommand{\BLACK}[1]{\textcolor{black}{#1}}
\newcommand{\WHITE}[1]{\textcolor{WHITE}{#1}}
\newcommand{\MAGENTA}[1]{\textcolor{MAGENTA}{#1}}
\newcommand{\ORANGE}[1]{\textcolor{ORANGE}{#1}}
\newcommand{\BLUE}[1]{\textcolor{BLUE}{#1}}
\newcommand{\GRAY}[1]{\textcolor{gray}{#1}}
\newcommand{\CYAN}[1]{\textcolor{cyan }{#1}}

\newcommand{\reference}[2]{\textcolor{PINK}{[#1~#2]}}
%\newcommand{\vect}[1]{\stackrel{\rightarrow}{#1}}

% Use some nice templates
\beamertemplatetransparentcovereddynamic

\newcommand{\A}{{\mathbb A}}
\newcommand{\degr}{\mathrm{deg}}

\title{Creating a \commonlisp{} implementation\\(Part 1)}

\author{Robert Strandh}
\institute{
}
\date{June, 2020}

%\inputtex{macros.tex}

\begin{document}
\frame{
\titlepage
}

\setbeamertemplate{footline}{
\vspace{-1em}
\hspace*{1ex}{~} \GRAY{\insertframenumber/\inserttotalframenumber}
}

\frame{
\frametitle{Compiler for a traditional batch language}
\vskip 0.25cm
\begin{figure}
\begin{center}
\inputfig{fig-c-unix-1.pdf_t}
\end{center}
\end{figure}
}

\frame{
\frametitle{Compiler for a traditional batch language}
\vskip 0.25cm
Characteristics:
\vskip 0.25cm
\begin{itemize}
\item Macros and declarations (implicit or explicit) are entered into
  the environment.
\item The compiler uses the environment to emit warnings, and to
  determine how to generate code.
\end{itemize}
}

\frame{
\frametitle{Run-time support for a traditional batch language}
\vskip 0.25cm
\begin{figure}
\begin{center}
\inputfig{fig-c-unix-2.pdf_t}
\end{center}
\end{figure}
}

\frame{
\frametitle{Run-time support for a traditional batch language}
\vskip 0.25cm
Characteristics:
\vskip 0.25cm
\begin{itemize}
\item Each program executes in a separate address space
\item Systems calls are used for file I/O, communication between
  programs, configuration, etc.
\item Communication between programs uses pipes, requiring transitions
  through the kernel.
\end{itemize}
}

\frame{
\frametitle{\commonlisp{} file compiler}
\vskip 0.25cm
\begin{figure}
\begin{center}
\inputfig{fig-common-lisp-file-compilation.pdf_t}
\end{center}
\end{figure}
}

\frame{
\frametitle{\commonlisp{} file compiler}
\vskip 0.25cm
Characteristics:
\vskip 0.25cm
\begin{itemize}
\item Requires and existing \commonlisp{} implementation (at least partial).
\item more?
\end{itemize}
}

\frame{
\frametitle{Run-time support for \commonlisp{} functions}
\vskip 0.25cm
\begin{figure}
\begin{center}
\inputfig{fig-common-lisp-run-time-support.pdf_t}
\end{center}
\end{figure}
}

\frame{
\frametitle{Run-time support for \commonlisp{} functions}
\vskip 0.25cm
Characteristics:
\vskip 0.25cm
\begin{itemize}
\item All functions share a common address space.
\item Functions call other functions directly.
\item Arbitrary data structures can be passed as arguments.
\end{itemize}
}

\frame{
\frametitle{Creating a \commonlisp{} implementation}
\vskip 0.25cm
\begin{itemize}
\item A compiler for a traditional language is a ``simple'' file
  translator.
\item Run-time support for a traditional language is provided by the
  operating-system kernel.
\item A compiler for a \commonlisp{} system is a bit more involved.
\item Run-time support for \commonlisp{} is the \commonlisp{} system.
\end{itemize}
\vskip 0.25cm
Creating a \commonlisp{} implementation involves writing a compiler, but
also creating the run-time support, which has some of the aspects of
a traditional operating-system kernel.
}

\frame{
\frametitle{General assumptions}
\vskip 0.25cm
\begin{itemize}
\item We want to use \commonlisp{} as much as possible.
\item The resulting system should not be too slow, but we do not
  need extremely good performance.
\item No classes and no generic functions.  This assumption will be
  revisited later.
\end{itemize}
}

\frame{
\frametitle{Strategy 1: Start with a small core in (say) C}
\vskip 0.25cm
\begin{itemize}
\item Write a minimal subset in an existing language.
\item Add more and more functionality, written in \commonlisp{}.
\end{itemize}
}

\frame{
\frametitle{Strategy 1: Core functionality}
\vskip 0.25cm
We need to figure out what initial functionality the core must have.
\vskip 0.25cm
\begin{itemize}
\item A memory manager and garbage collector.
\item A reader.  It must be possible to read additional \commonlisp{} code.
\item An evaluator.  The additional \commonlisp{} code must be executed.
\item A printer, i.e., the \texttt{print} function of \commonlisp{}.
\end{itemize}
}

\frame{
\frametitle{Strategy 1: Memory manager and garbage collector}
\vskip 0.25cm
A natural choice:
\begin{itemize}
\item Use C malloc() to allocate \commonlisp{} objects.
\item Use the Boehm-Demers-Weiser conservative
garbage collector to reclaim memory of dead objects.
\end{itemize}
}

\frame{
\frametitle{Strategy 1: Reader}
\vskip 0.25cm
The \commonlisp{} reader is a complicated module, and is best written
in \commonlisp{}, but that's not a choice for the core.  Two options:
\begin{itemize}
\item Write a subset of the reader in C, capable of reading additional
  source code.  Replace with a full reader written in \commonlisp{}
  later.
\item Write the final reader in C, but leave out complicated standard
  reader macros that can be written in \commonlisp{} later.
\end{itemize}
}

\frame{
\frametitle{Strategy 1: Printer}
\vskip 0.25cm
Like the reader, the \commonlisp{} printer is a complicated module,
and is best written in \commonlisp{}, but that's not a choice for the
core.
\vskip 0.25cm
The same two options are possible.  We will not discuss the printer
any further.
}

\frame{
\frametitle{Strategy 1: Evaluator}
\vskip 0.25cm
Several possible implementations:
\vskip 0.25cm
\begin{itemize}
\item A direct interpreter written in C.
\item A compiler generating native machine code.
\item A compiler generating byte codes, combined with a byte-code
  interpreter written in C.
\end{itemize}
}

\frame{
\frametitle{Strategy 1: Direct interpreter}
\vskip 0.25cm
\begin{itemize}
\item Relatively simple.
\item Slow.
\item ``Cross evaluation'' is not possible.
  interpreter written in C.
\end{itemize}
}

\frame{
\frametitle{Strategy 1: Compiler generating native code}
\vskip 0.25cm
\begin{itemize}
\item Hard to write.
\item Requires knowledge of the C ABI.
\item Makes tail-call optimization somewhat difficult.
\end{itemize}
}

\frame{
\frametitle{Strategy 1: Compiler generating byte codes}
\vskip 0.25cm
\begin{itemize}
\item Relatively simple to write.
\item Reasonably fast.
\item Tail-call optimization is relatively easy.
\end{itemize}
This is our recommended choice, at least for the time being.
}

\frame[containsverbatim]{
\frametitle{Strategy 1: Complications}
\vskip 0.25cm
\commonlisp{} does not have a unique set of basic operators.  There
are many possible choices.
\vskip 0.25cm
The following link is to a page with a long list of possible choices:
http://home.pipeline.com/~hbaker1/MetaCircular.html
}

\frame[containsverbatim]{
\frametitle{Strategy 1: Complications}
\vskip 0.25cm
Perhaps we would like to implement \texttt{mapcar} in \commonlisp{}.
Here is a reasonable-looking (simplified) implementation:
\vskip 0.25cm
\begin{verbatim}
(defun mapcar (function list)
  (do ((sublist list (rest sublist))
       (result '()))
      ((null sublist) (nreverse result))
    (push (funcall function (first sublist))
          result)))
\end{verbatim}
\vskip 0.25cm
But for this implementation to work, the \texttt{do} macro must exist.
\vskip 0.25cm
And the macro expander for \texttt{do} may very well use
\texttt{mapcar} to extract the local variables, here \texttt{sublist}
and \texttt{result}.
}

\frame[containsverbatim]{
\frametitle{Strategy 1: Complications}
\vskip 0.25cm
Possible solutions:
\vskip 0.25cm
\begin{itemize}
\item Write \texttt{mapcar} in the implementation language of the
  core (C), but this solution goes against our goals to use
  \commonlisp{} as much as possible.
\item Use a simpler iteration construct, hoping its expansion does not
  require \texttt{mapcar}.
\item Use \texttt{tagbody} and \texttt{go} to implement \texttt{mapcar}.
\end{itemize}
\vskip 0.25cm
A similar decision has to be made for almost all non-trivial function
to be implemented.
\vskip 0.25cm
Worse, during maintenance, these choices have to be known so as to
avoid violations of the constraints.
}

\frame{
\frametitle{Thank you}
}

\end{document}
